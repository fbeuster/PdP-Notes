	\subsection{Übung 1} % (fold)
	\label{sub:uebung_1}
	Die behandelten Shell-Programme und Funktionen sind meist nur lose augelistet. Genaue Informationen
	lassen sich über die Man-Pages erfahren.

	\lstShell[print working directory]
	\begin{lstlisting}
$ pwd
	\end{lstlisting}

	\lstShell[Weitere Shell-Programme zur Dateiverwaltung]
	\begin{lstlisting}
cd, ls, exit, mkdir, rm, touch, cp, mv
	\end{lstlisting}

	\subsubsection*{Verzeichnisse im Unix-Dateisystem} % (fold)
	\label{ssub:verzeichnisse_im_unix_dateisystem}

	\begin{tabular}{ll}
		\texttt{/tmp} 			& Temporär\\
		\texttt{/dev} 			& devices\\
		\texttt{/usr} 			& unix system resources\\
		\texttt{(usr/local)} 	& Programme (nicht immer vorhanden)\\
		\texttt{/bin} 			& Binärdaten\\
		\texttt{/home} 			& \multirow{2}{*}{User data}\\
		\texttt{/users} 		&\\
		\texttt{/dev/hda}		& \multirow{2}{*}{Drives}\\
		\texttt{/dev/sda}		&\\
		\texttt{dev/tty}		& Terminals\\
		\texttt{dev/null}		& Garbage
	\end{tabular}

	% subsubsection verzeichnisse_im_unix_dateisystem (end)

	\subsubsection*{Verzeichnisnavigation} % (fold)
	\label{ssub:verzeichnisnavigation}

	Pfade, die mit einem \texttt{/} beginnen, sind absolut, alle anderen sind relativ.
	
	\lstShell
	\begin{lstlisting}
$ cd ~  # home directory
$ cd -  # go back
$ cd .. # go up
	\end{lstlisting}

	% subsubsection verzeichnisnavigation (end)

	\lstShell[Aufruf Man-Page (Handbuch) von \texttt{myCommand}]
	\begin{lstlisting}
$ man myCommand
	\end{lstlisting}

	\lstShell[Suche nach \texttt{term}]
	\begin{lstlisting}
$ apropos term
	\end{lstlisting}

	\lstShell[word count in \texttt{file} (auch mehr als Worte)]
	\begin{lstlisting}
$ wc file
	\end{lstlisting}

	\subsubsection*{Dateiberechtigungen} % (fold)
	\label{ssub:dateiberechtigungen}
	
	\begin{tabular}{lllll}
		\texttt{d} 	& \texttt{rwx} 	& \texttt{rwx} 	& \texttt{rwx} 	& \texttt{n}\\
		Typ			& user 			& group 		& other 		& Verzeichnisanzahl
	\end{tabular}\\

	Dabei kann der Typ folgendes sein:

	\begin{tabular}{ll}
		\texttt{d} 	& Directory\\
		\texttt{l} 	& Link\\
		\texttt{p} 	& named Pipe\\
		\texttt{-} 	& File\\
		\texttt{c} 	& Character Device\\
		\texttt{b} 	& Block Device
	\end{tabular}

	% subsubsection dateiberechtigungen (end)

	\subsubsection*{Dateilinks erstellen} % (fold)
	\label{ssub:dateilinks_erstellen}
	
	\begin{figure}[hp]
		\caption{Hartlink über \texttt{ln}}
		\includegraphics[width=\textwidth]{workfiles/u1_1}
	\end{figure}
	
	\begin{figure}[hp]
		\caption{Softlink über \texttt{ln -s}}
		\includegraphics[width=\textwidth]{workfiles/u1_2}
	\end{figure}

	Löscht man \texttt{Original} kann über \texttt{Kopie} noch auf die Datei zugegriffen werden, ein
	Aufruf von \texttt{Softlink} führt allerdings zu Fehlern.

	% subsubsection dateilinks_erstellen (end)

	\subsubsection*{Dateisysteme} % (fold)
	\label{ssub:dateisysteme}
	
	\lstShell[Ein-/Aushängen von Dateisystemen]
	\begin{lstlisting}
$ mount
$ umount
	\end{lstlisting}

	\lstShell[Speicherverbrauch anzeigen]
	\begin{lstlisting}
$ du	# Festplattenplatzverbrauch in Byte
$ df	# liefert zusaetzliche Details
	\end{lstlisting}

	\lstShell[Ändern von Dateiberechtigungen]
	\begin{lstlisting}
$ chmod u=rwx file
$ chmod u-r file
$ chmod u+r file
$ chmod 600 file
	\end{lstlisting}
	
	Bei obiger Darstellung ist \texttt{u} die Kennzeichnung für den aktuellen User, möglich ist auch
	\texttt{g} (Group), \texttt{o} (other) oder \texttt{a} (alle).

	Per Zuweisungen können alle	Permutationen von \texttt{rwx} vergeben werden. Oder per \texttt{-/+} einzelne
	Rechte bearbeitet werden.

	Eine direkte Angaabe als Zahlwert ist ebenfalls möglich. Für jede Nutzerklasse bildet man die Summe
	aus read (4), write (2) und execute (1).

	\lstShell[Ändern von Dateibesitzern]
	\begin{lstlisting}
$ chown user file
$ chown user:group file
	\end{lstlisting}

	% subsubsection dateisysteme (end)

	\subsubsection*{Wildcards in der Shell} % (fold)
	\label{ssub:wildcards_in_der_shell}

	\begin{tabular}{ll}
		\texttt{*} 	& beliebige Zeichenfolge\\
		\texttt{?} 	& beliebiges Zeichen\\
		\texttt{[abcD]} 	& \multirow{2}{*}{\underline{ein} Zeichen aus der Gruppe}\\
		\texttt{[a-z]} 	&\\
		\texttt{\{a,b\}} 	& Aufzählung von Pattern\\
	\end{tabular}

	% subsubsection wildcards_in_der_shell (end)

	% subsection uebung_1 (end)