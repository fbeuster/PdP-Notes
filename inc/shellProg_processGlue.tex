

	\subsection{Prozesspipelines und Glue Languages} % (fold)
	\label{sub:prozesspipelines_und_glue_languages}
	
	Unix verfolgt das Konzept: \textbf{Alles ist eine Datei}
	\begin{flalign*}
		 \left.\begin{array}{c}
		 	\text{.c}\\
		 	\text{.java}\\
		 	\text{.f77}
		 \end{array}\right\}\text{ Compiler }\rightarrow\text{a.out}
	\end{flalign*}
	Vorteile:
	\begin{itemize}
		\item auch Kompilat braucht keine unterschiedlichen Prozeduren
		\item kein gerätespezifischer Code
	\end{itemize}
	Dateiunabhängig ist damit möglich
	\lstCcode[Lies aus Datei 1 1 Byte und speicher es in c]
	\begin{lstlisting}
read(1,1,&c);
	\end{lstlisting}

	\begin{figure}[hb]
		\caption{Schnittstellen auf System-Ebene zu Abstraktion}
		\includegraphics[width=\textwidth]{workfiles/v2_1}
	\end{figure}

	\newpage

	\begin{figure}[h]
		\caption{Hierarchisches Dateisystem}
		\begin{center}
			\tikzstyle{every node}=[anchor=west]
			\begin{tikzpicture}[%
			  grow via three points={one child at (0.5,-0.7) and
			  two children at (0.5,-0.7) and (0.5,-1.4)},
			  edge from parent path={(\tikzparentnode.south) |- (\tikzchildnode.west)}]
			  \node {/}
			    child { node {a}}
			    child { node {b}}
			    child { node {c/}
			      child { node {x}}
			      child { node {y}}
			      child { node {z/}}
			    };
			\end{tikzpicture}
		\end{center}
	\end{figure}

	\begin{figure}[h]
		\caption{Kommunikation zwischen Kernel und Prozess}
		\includegraphics[width=\textwidth]{workfiles/v2_2}
	\end{figure}
	Auch die Shell ist ein Prozess und folgt diesem Kommunikationsschema.\\

	\subsubsection*{Lokaler Dateiaufruf} % (fold)
	\label{ssub:dateiaufrufe}
		
		\lstCcode[Lokaler Dateiufruf]
		\begin{lstlisting}
File *fopen(path)
		\end{lstlisting}

		\begin{figure}[h]
			\caption{Dateiaufruf im Programm}
			\includegraphics[width=\textwidth]{workfiles/v2_3}
		\end{figure}

		Merkmale des Konzepts
		\begin{itemize}
			\item lokale Aufrufe schneller
			\item andere Prozesse können Datei überschreiben
		\end{itemize}

		\lstCcode[Aufbau einer \texttt{FILE}-Struktur nach \texttt{fopen()}]
		\begin{lstlisting}
f = fopen(path)
f->buf	// Speicher
f->len	// Laenge
f->fd		// File Descriptor
		\end{lstlisting}

	% subsubsection dateiaufrufe (end)

	\subsubsection*{Systemweiter Dateiaufruf} % (fold)
	\label{ssub:systemweiter_dateiaufruf}

		\lstCcode[Systemaufruf]
		\begin{lstlisting}
int open(path)
		\end{lstlisting}

		\begin{itemize}
			\item Systemaufruf, Kernel wird angesprochen
			\item verschiedene Prozesse können nicht gleichzeitig auf das 1. Byte zugreifen
		\end{itemize}
	
	% subsubsection systemweiter_dateiaufruf (end)

	\subsubsection*{Fork und Exec} % (fold)
	\label{ssub:aufruf_von_fork}
	
		\lstCcode[Welcher Prozess bin ich nach dem \texttt{fork()}?]
		\begin{lstlisting}
if(pid = fork()) {
 // parent
} else {
 // child
}
		\end{lstlisting}

		\lstCcode[Weiterführen mit \texttt{exec()}]
		\begin{lstlisting}
if(pid = fork()) {
 // parent
} else {
 // child
 exec();
}
		\end{lstlisting}

	% subsubsection aufruf_von_fork (end)

	\subsubsection*{Fork und Exec am Beispiel von butfirst.sh} % (fold)
	\label{ssub:fork_am_beispiel_von_butfirst_sh}
	
		\lstShell
		\begin{lstlisting}
#!/bin/sh
read
exec doit
		\end{lstlisting}

		\begin{figure}[th]
			\caption{Ablauf von Fork und Exec}
			\includegraphics[width=\textwidth]{workfiles/v2_4}
		\end{figure}

	% subsubsection fork_am_beispiel_von_butfirst_sh (end)

	% subsection prozesspipelines_und_glue_languages (end)

	%%%%
	%%%% end V2
	%%%%

	\subsubsection*{Textausgabe in Haskell} % (fold)
	\label{ssub:textausgabe_in_haskell}
	
		\lstHaskell
		\begin{lstlisting}
main = putstrln "Hello World!"
		\end{lstlisting}

		\lstHaskell[Mehrzeilige Ausgabe, Variante 1]
		\begin{lstlisting}
main = putstrln "Hello World!" >> putstrln "Foobar"
		\end{lstlisting}

		\lstHaskell[Mehrzeilige Ausgabe, Variante 2]
		\begin{lstlisting}
main = do {
		putstrln "Hello World!"
		putstrln "Foobar"
	}
		\end{lstlisting}

	% subsubsection textausgabe_in_haskell (end)

	\subsubsection*{Umgebungsvariablen} % (fold)
	\label{ssub:umgebungsvariablen}

		Ausgabe von \texttt{Hello\$USER} ist klar, aber was ist mit \texttt{\$USERHello}?
		\begin{itemize}
			\item gesucht wird nach \texttt{\$USERHello} $\rightarrow$ führt meist zu einem \texttt{not found}
			\item Lösung durch Klammerung: \texttt{\$\{USER\}Hello}
		\end{itemize}

	% subsubsection umgebungsvariablen (end)

	\subsubsection*{Trennzeichen in der Shell} % (fold)
	\label{ssub:trennzeichen_in_der_shell}
	
		Trennzeichen, die in der Varaible \texttt{\$ISF} stehen, werden zur Worttrennung genutzt:
		\lstShell
		\begin{lstlisting}
" \n\t"
		\end{lstlisting}
		Einfaches Ausgeben via \texttt{echo} führt allerdings zu keiner Ausgabe:
		\lstShell
		\begin{lstlisting}
$ echo $IFS
>
		\end{lstlisting}
		Grund: Es sind nur Whitespace-Zeichen, die abgeschnitten werden. Ausgabe als Whitespace möglich (erkennbar an zwei Newlines) über:
		\lstShell
		\begin{lstlisting}
$ echo "$IFS"
>
>
		\end{lstlisting}

	% subsubsection trennzeichen_in_der_shell (end)

	\subsubsection*{Shellübersicht} % (fold)
	\label{ssub:shell_bersicht}
	
		\begin{tabular}{lll}
			bash	&	Bourne Again Shell 	&	in Vorlesung genutzt\\
			sh		&	Bourne Shell 		&	ursrüngliche Shell\\
			csh		&	c Shell 			&	\\
			ksh		&	Korn Shell 			&	
		\end{tabular}

	% subsubsection shell_bersicht (end)

	\subsubsection*{True und False in der Shell} % (fold)
	\label{ssub:true_und_false_in_der_shell}
	
		In der Shell sind auch \texttt{true} und \texttt{false} ausführbare Programme.
		\lstCcode[\texttt{true} als C-Programm]
		\begin{lstlisting}
int main() { return 0; }
		\end{lstlisting}

	% subsubsection true_und_false_in_der_shell (end)

	\subsubsection*{Vorsicht vor deprecated Befehlen} % (fold)
	\label{ssub:vorsicht_vor_deprecated_befehlen}
	
		Veraltete Befehle können Sicherheitsrisiken sein. Beispiel \texttt{gets()}:

		\lstCcode[testpwd.c]
		\begin{lstlisting}
getInp() {
	char buff[1000];
	gets(buff);
}
		\end{lstlisting}
		\begin{figure}[hbtp]
			\caption{Illustrations des gets-Problems}
			\includegraphics[width=\textwidth]{workfiles/v3_1}
		\end{figure}

	% subsubsection vorsicht_vor_deprecated_befehlen (end)

	\subsubsection*{Dateiumleitungen} % (fold)
	\label{ssub:dateiumleitungen}
	
		\begin{tabular}{ll}
			\texttt{> file}		&	in \texttt{file} leiten	\\
			\texttt{>> file}	&	an \texttt{file} anfügen	\\
			\texttt{< file}		&	aus \texttt{file} lesen	\\
			\texttt{1 > file}	&	File Descriptor 1 leitet in \texttt{file}	\\
			\texttt{2 >\& 1}	&	FD2 leitet in selbes Ziel wie FD 1	\\
			\texttt{FD0}		&	Standardinput	\\
			\texttt{FD1}		&	Standardoutput	\\
			\texttt{FD2}		&	Standarfehlerausgabe	\\
		\end{tabular}

		\begin{flalign*}
			&\text{\texttt{printf("Hello$\backslash$n");}}	\\
			&\qquad\downarrow	\\
			&\text{\texttt{FILE *stdout}}	\\
			&\qquad\downarrow	\\
			&\text{\texttt{write(stdout->fd, stdout->buf, stdout->len);}}
		\end{flalign*}

	% subsubsection dateiumleitungen (end)

	%%%%
	%%%% end V3
	%%%%