

	Themen der Vorlesung:
	\begin{itemize}
		\item Shellprogrammierung
		\item Standardtools
		\item Fallstudie DZNE sensor based monitoring
	\end{itemize}

	\subsection{Shells und Tools} % (fold)
	\label{sub:shells_und_tools}

		\texttt{numbers.dat} enthält $793\cdot10^8$ Zahlen, \texttt{summary} rechnet Werte
		spaltenweise aus

		
		\lstShell[Einfache Eingabeumleitung]
		\begin{lstlisting}
$ cp summary < numbers.dat
		\end{lstlisting}

		\lstShell[1. Zeile auslassen, da bspw. Kommentar]
		\begin{lstlisting}
$ tail -1 < numbers.dat | summary
		\end{lstlisting}

		Problem bei \texttt{tail} und anderen Pipeline-Operationen:\\
		Es laufen 2 Programme,  alle Daten durchlaufen die CPU zweimal, problematisch auf
		Single Core\\

		Shell-Skript (siehe Foliensatz) besser geeignet. Aufruf dann wieder:

		\lstShell
		\begin{lstlisting}
$ buttfirst.sh < numbers.dat
		\end{lstlisting}

		Eine gesonderte Berücksichtigung der Standardeingabe/-ausgabe muss nicht durch das
		Programm erfolgen

		\subsubsection*{Fallstudie: DZNE sensor based monitoring} % (fold)
		\label{ssub:fallstudie_dzne_sensor_based_monitoring}
		% subsubsection fallstudie_dzne_sensor_based_monitoring (end)
	
	% subsection shells_und_tools (end)

	%%%%
	%%%% end V1
	%%%%