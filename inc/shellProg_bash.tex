	\subsection{Bash-Skripte und weitere Tools} % (fold)
	\label{sub:bash_skripte_und_weitere_tools}

		\lstBash[Rückgabewert des letzten Programms]
		\begin{lstlisting}
echo $?
		\end{lstlisting}
		\lstBash[direkte Ausgabe bis zum nächsten \texttt{eof}]
		\begin{lstlisting}
cat << 'eof'
		\end{lstlisting}
		\lstBash[erst Auswertung, dann Ausgabe]
		\begin{lstlisting}
cat << eof
		\end{lstlisting}
		Technik bereits vom Interpreter bekannt:
		\lstBash
		\begin{lstlisting}
cat > testing.c << eof
# C-Code mit Expressions
eof
		\end{lstlisting}

		\subsubsection*{Kontrollmechanismen} % (fold)
		\label{ssub:kontrollmechanismen}
		
			\lstBash[Einfache Verzeweigungen]
			\begin{lstlisting}
if <cmd> then
	<cmd>
elif <cmd> then
	<cmd>
else
	<cmd>
fi
			\end{lstlisting}
			\lstBash[\texttt{case}-Verzweigung]
			\begin{lstlisting}
case <var> in
	(Muster) <cmd> ;;
	(Muster) <cmd> ;;
esac
			\end{lstlisting}
			Hierbei ist als \texttt{Muster} eine vereinfachte Form von RegEx möglich.
			\lstBash[\texttt{while}-Schleife]
			\begin{lstlisting}
while <cmd>
do
	<cmd>
done
			\end{lstlisting}
			\lstBash[\texttt{unitl}-Schleife]
			\begin{lstlisting}
until <cmd>
do
	<cmd>
done
			\end{lstlisting}
			\lstBash[\texttt{for}-Schleife]
			\begin{lstlisting}
for <var> in w1 w2 w3 ...
do
	<cmd>
done
			\end{lstlisting}
		% subsubsection kontrollmechanismen (end)

		\subsubsection*{Beispiele Kontrollmechanismen} % (fold)
		\label{ssub:beispiele_kontrollmechanismen}

			\lstBash[Ausgabe Verzeichnisinhalt]
			\begin{lstlisting}
	for i in *; do echo $i; done
			\end{lstlisting}
			\lstBash
			\begin{lstlisting}
	for((i=1;i<200;i++)) do ... done
			\end{lstlisting}
			\lstBash[erstellt einfaches Auswahlmenü mit Verzeichnisinhalt]
			\begin{lstlisting}
	select i in *; do echo $i; done
			\end{lstlisting}
		
		% subsubsection beispiele_kontrollmechanismen (end)

		\subsubsection*{Funktionen und Argumente} % (fold)
		\label{ssub:funktionen_und_argumente}
		
			\lstBash[Variablenklammer bleibt leer]
			\begin{lstlisting}
func() {
	...
}
			\end{lstlisting}
			\lstBash
			\begin{lstlisting}
$0 $1 $2 ...
			\end{lstlisting}
			Wird ein \texttt{shift} auf den Variablen ausgeführt, ergibt sich folgendes Schema\\
			\begin{tikzpicture}
				\matrix (m) [matrix of math nodes,row sep=1.5em,column sep=2em,minimum width=2em]
				{
					\texttt{\$0} & \texttt{\$1} & \texttt{\$2} & ... & & \texttt{\$8} & \texttt{\$9} \\
					\texttt{\$0} & \texttt{\$1} &  & ... & \texttt{\$7} & \texttt{\$8} & \\
				};
				\path[->]
					(m-1-1) edge (m-2-1)
					(m-1-3) edge (m-2-2)
					(m-1-6) edge (m-2-5)
					(m-1-7) edge (m-2-6);
			\end{tikzpicture}
(\$0 - Name des Programms
		% subsubsection funktionen_und_argumente (end)

		\subsubsection*{Textverarbeitung} % (fold)
		\label{ssub:textverarbeitung}
		
		\lstShell[Werte zuweisen mittels \texttt{read}]
		\begin{lstlisting}
$ read a b c
> 1 2 3 4 5
		\end{lstlisting}
		\lstShell[Ausgabe gespeicherter Werte]
		\begin{lstlisting}
$ echo $a $b -- $c
> 1 2 -- 3 4 5
		\end{lstlisting}
		\lstShell[Formatierte Strings]
		\begin{lstlisting}
$ printf "%f\n" 1e-6
		\end{lstlisting}
		% subsubsection textverarbeitung (end)

		\subsubsection*{Streameditor \texttt{sed}} % (fold)
		\label{ssub:streameditor_sed}
		
		\lstShell[liefert Zeile 1 und 1000]
		\begin{lstlisting}
$ sed -n -e'1p -e'1000p' < data.csv
		\end{lstlisting}
		\texttt{sed} ist sehr leistungsfähig für die Zielenverabeitung von Daten

		% subsubsection streameditor_sed (end)

		\subsubsection*{Aho-Weinberg-Kernigton \texttt{awk}} % (fold)
		\label{ssub:aho_weinberg_kernigton_awk}
		
		\lstShell
		\begin{lstlisting}
awk  '/Muster/{Anweisung}'
awk '/^\#/{print $2}'
awk '{print $2}'      # Ausfuehrung immer
awk 'NR==3{...}'       # Zeile 3
awk 'NR<2&&/^\#/{...}'  # vor Zeile 2 und das Muster treffend
		\end{lstlisting}

	%%%%
	%%%% end V5
	%%%%
	%%%% V6
	%%%%

			Im Allgemeinen exisitiert in Unix folgende Vorstellung der Datenverarbeitung:\\

			% Define block styles
			\tikzstyle{box} = [rectangle, draw, text centered, minimum height=2em]
			\tikzstyle{emp} = [rectangle, text centered, minimum height=2em]
			\tikzstyle{line} = [draw, -latex']
			\begin{tikzpicture}[node distance = 1.5cm, auto]
			    % Place nodes
			    \node [emp] (0a) {};
			    \node [emp, right of=0a] (0b) {};
			    \node [box, above right of=0b] (1) {};
			    \node [box, below right of=0b] (2) {};
			    \node [box, right of=1] (3) {};
			    \node [box, below right of=3] (4) {};
			    \node [box, right of=4] (5) {awk};
			    \node [box, right of=5] (6) {};
			    \node [emp, right of=6] (f) {};
				
			    % Draw edges
				\path [line] (0a) -- (0b);
				\path [line] (0b) -- (1);
				\path [line] (0b) -- (2);
				\path [line] (1) -- (3);
				\path [line] (2) -- (4);
				\path [line] (3) -- (4);
				\path [line] (4) -- (5);
				\path [line] (5) -- (6);
				\path [line] (6) -- (f);
			\end{tikzpicture}

			\lstBash[awk-File]
			\begin{lstlisting}
pattern {action} # jede Eingabezeile wird von oben
pattern {action} # nach unten geprueft
...
			\end{lstlisting}

		% subsubsection aho_weinberg_kernigton_awk (end)

		\subsubsection*{Weiterer Bezug auf das Sensorbeispiel} % (fold)
		\label{ssub:weiterer_bezug_auf_das_sensorbeispiel}

			\begin{flalign*}
				 \left.\begin{array}{rcl}
				 	-2g & \approx & 0\\
				 	0g & \approx & 2047\\
				 	2g & \approx & 4095\\
				 \end{array}\right\}x_{raw}\\
				 \rightarrow\text{12bit Sensorauflösung}
			\end{flalign*}
			Gewünscht ist dafür eine Zentrierung
			\begin{flalign*}
				x_{cooked}\sim N\left(\mu=0,\sigma=1\right)
			\end{flalign*}
			Dafür werden folgende Werte erfasst

			\begin{flalign*}
				(Abweichung / Summe aller Werte)\\
				\text{\texttt{x[i]}}=\sum_{j=1}^nx_j\\
				(Varianz / Summe der Quadrate)\\
				\text{\texttt{xx[i]}}=\sum_{j=1}^nx_j^2
			\end{flalign*}
			Wie realisiert \texttt{awk} Felder? Balancierte Bäume.\\
			\texttt{x[i]} - i ist Index des Baumblattes\\
			\texttt{x[i,j]} - i\^{}@j ist Index des Blattes\\

			Vorsicht: Trennzeichen sollte nicht in Indexstring vorkommen. Dieser wird benutzt um den Wunsch nach mehrdimensionalen Feldern zu erfüllen: anstelle von i=7 und j=115 wird der String 7\^{}@115 als Schlüssel benutzt.\\

			\lstShell[Anwendung von \texttt{prog} auf übergebene Dateien]
			\begin{lstlisting}
$ awk -f prog file1 file2
			\end{lstlisting}
			Hierbei wird jede Datei zeilenweise verarbeitet. Für das Programm existieren
			zwei Zähler:\\
			NR - aktuelle Zeilennummer insgesamt\\
			FNR - aktuelle Zeilennummer in einer Datei\\
			NF - Anzahl an Feldern des aktuellen Satzes

			Formeln für \texttt{rescale.awk}
			\begin{flalign*}
				x &= \frac{x_{raw}-\mu}{\sigma}\\
				r &= \sqrt{x^2+y^2+z^2}
			\end{flalign*}

			Summieren von großen Zahlenmengen ist bspw. durch Kahan-Summing gut lösbar.

		
		% subsubsection weiterer_bezug_auf_das_sensorbeispiel (end)
	
	% subsection bash_skripte_und_weitere_tools (end)

	%%%%
	%%%% end V6
	%%%%