%%%
%%% V10
%%%

	Alte Sprachen, die auch heute noch in Verwendung sind, sind vor allem \textbf{Lisp} und
	\textbf{COBOL}. Neuere Sprachen, wie bspw. \textbf{Swift}, haben Züge von Lisp.
	\begin{flalign*}
		 \left.\begin{array}{c}
		 	\text{Emacs}\\
		 	\text{AutoCAD}\\
		 	\text{InterleatTPS}\\
		 	\text{R}
		 \end{array}\right\}\text{ bauen auf Lisp auf}
	\end{flalign*}

	%%% snippet V11

	\subsection{Eigenschaften von Lisp} % (fold)
	\label{sub:eigenschaften_von_lisp}
	
	\begin{itemize}
		\item stateful
		\item dynamic typing
		\item program as data
		\item s-expressions, z.B. \texttt{(+ 3 4 5)}\\
		\begin{tikzpicture}
			\node {\texttt{+}}
				child { node {\texttt{3}}}
				child { node {\texttt{4}}}
				child { node {\texttt{5}}};
		\end{tikzpicture}
	\end{itemize}

	% subsection eigenschaften_von_lisp (end)

	%%% snippet V11

	\lstLisp[Auswertung des Ausdrucks durch \texttt{'} verhindern]
	\begin{lstlisting}
(quote (1 2 3)) = '(1 2 3)
	\end{lstlisting}

	\subsection{Listen-Konzept} % (fold)
	\label{sub:listen_konzept}
	
		Lisp-Programme sind Listen von Symbolen und Ausdrücken.

		\lstLisp[Liste in Lisp]
		\begin{lstlisting}
(1 2)
		\end{lstlisting}

		\lstHaskell[Liste in Haskell]
		\begin{lstlisting}
[1,2] == 1:2:[]
		\end{lstlisting}

		\begin{figure}[H]
			\centering
			\caption{Veranschaulichung des vorherigen Listenbeispieles}
			\vspace*{-15pt}
			\rotatebox{-90}{
				\tikzstyle{cell} = [draw,ellipse,ellipse split]
				\tikzstyle{leaf} = [draw,circle]
				\begin{tikzpicture}
				  \node (1) at (0,0)  [cell]{};
				  \node (2) at (1,-1) [leaf]{\rotatebox{90}{1}};
				  \node (3) at (1,1)  [cell]{};
				  \node (4) at (2,0)  [leaf]{\rotatebox{90}{2}};
				  \node (5) at (2,2)  {};
				  \path (1) edge [->,thick,out=-45,in=180] (2);
				  \path (1) edge [->,thick,out=45,in=180] (3);
				  \path (3) edge [->,thick,out=-45,in=180] (4);
				  \path (3) edge [-|,thick,out=45,in=180] (5);
				\end{tikzpicture}
			}
		\end{figure}
		\begin{figure}[H]
			\centering
			\caption{Veranschaulichung Listenelement allgemein}
			\vspace*{-15pt}
			\rotatebox{-90}{
				\tikzstyle{cell} = [draw,ellipse,ellipse split]
				\begin{tikzpicture}
				  \node (1) at (0,0)  [cell]{};
				  \node (2) at (-1,-1) {\rotatebox{90}{car}};
				  \node (3) at (-1,1)  {\rotatebox{90}{cdr}};
				  \path (1) edge [thick,out=225,in=0] (2);
				  \path (1) edge [thick,out=135,in=0] (3);
				\end{tikzpicture}
			}
		\end{figure}
		\begin{figure}[H]
			\centering
			\caption{1-elementige Liste \texttt{(1)}}
			\vspace*{-15pt}
			\rotatebox{-90}{
				\tikzstyle{cell} = [draw,ellipse,ellipse split]
				\tikzstyle{leaf} = [draw,circle]
				\begin{tikzpicture}
				  \node (1) at (0,0)  [cell]{};
				  \node (2) at (1,-1) [leaf]{\rotatebox{90}{1}};
				  \node (3) at (1,1)  {};
				  \path (1) edge [->,thick,out=-45,in=180] (2);
				  \path (1) edge [->,thick,out=45,in=180] (3);
				\end{tikzpicture}
			}
		\end{figure}
		
		\begin{figure}[H]
			\centering
			\caption{Selbstreferenzen sind möglich}
			\vspace*{-40pt}
			\rotatebox{-90}{
				\tikzstyle{cell} = [draw,ellipse,ellipse split]
				\tikzstyle{leaf} = [draw,circle]
				\begin{tikzpicture}
				  \node (1) at (0,0)  [cell]{};
				  \node (2) at (1,-1) [leaf]{\rotatebox{90}{1}};
				  \path (1) edge [->,thick,out=-45,in=180] (2);
				  \draw (1) edge [->,thick,out=45,in=180,looseness=8] (1);
				\end{tikzpicture}
			}
		\end{figure}
		
		\begin{figure}[H]
			\centering
			\caption{\texttt{'\#1=(\#1\# . \#1\#)}}
			\vspace*{-35pt}
			\rotatebox{-90}{
				\tikzstyle{cell} = [draw,ellipse,ellipse split]
				\begin{tikzpicture}
				  \node (1) at (0,0)  [cell]{};
				  \draw (1) edge [->,thick,out=-45,in=180,looseness=8] (1);
				  \draw (1) edge [->,thick,out=45,in=180,looseness=8] (1);
				\end{tikzpicture}
			}
		\end{figure}
		
		\begin{figure}[H]
			\centering
			\caption{\texttt{'\#1=(1 2 3 . \#1\#)}}
			\vspace*{-70pt}
			\rotatebox{-90}{
				\tikzstyle{cell} = [draw,ellipse,ellipse split]
				\tikzstyle{leaf} = [draw,circle]
				\begin{tikzpicture}
				  \node (1) at (0,0)  [cell]{};
				  \node (2) at (1,-1) [leaf]{\rotatebox{90}{1}};
				  \node (3) at (1,1)  [cell]{};
				  \node (4) at (2,0) [leaf]{\rotatebox{90}{2}};
				  \node (5) at (2,2)  [cell]{};
				  \node (6) at (3,1) [leaf]{\rotatebox{90}{3}};
				  \path (1) edge [->,thick,out=-45,in=180] (2);
				  \path (1) edge [->,thick,out=45,in=180] (3);
				  \path (1) edge [->,thick,out=-45,in=180] (2);
				  \path (1) edge [->,thick,out=45,in=180] (3);
				  \path (3) edge [->,thick,out=-45,in=180] (4);
				  \path (3) edge [->,thick,out=45,in=180] (5);
				  \path (5) edge [->,thick,out=-45,in=180] (6);
				  \path (5) edge [->,thick,out=45,in=180,looseness=2] (1);
				\end{tikzpicture}
			}
		\end{figure}

	% subsection listen_konzept (end)